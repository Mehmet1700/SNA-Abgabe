% Options for packages loaded elsewhere
\PassOptionsToPackage{unicode}{hyperref}
\PassOptionsToPackage{hyphens}{url}
%
\documentclass[
]{article}
\usepackage{amsmath,amssymb}
\usepackage{iftex}
\ifPDFTeX
  \usepackage[T1]{fontenc}
  \usepackage[utf8]{inputenc}
  \usepackage{textcomp} % provide euro and other symbols
\else % if luatex or xetex
  \usepackage{unicode-math} % this also loads fontspec
  \defaultfontfeatures{Scale=MatchLowercase}
  \defaultfontfeatures[\rmfamily]{Ligatures=TeX,Scale=1}
\fi
\usepackage{lmodern}
\ifPDFTeX\else
  % xetex/luatex font selection
\fi
% Use upquote if available, for straight quotes in verbatim environments
\IfFileExists{upquote.sty}{\usepackage{upquote}}{}
\IfFileExists{microtype.sty}{% use microtype if available
  \usepackage[]{microtype}
  \UseMicrotypeSet[protrusion]{basicmath} % disable protrusion for tt fonts
}{}
\makeatletter
\@ifundefined{KOMAClassName}{% if non-KOMA class
  \IfFileExists{parskip.sty}{%
    \usepackage{parskip}
  }{% else
    \setlength{\parindent}{0pt}
    \setlength{\parskip}{6pt plus 2pt minus 1pt}}
}{% if KOMA class
  \KOMAoptions{parskip=half}}
\makeatother
\usepackage{xcolor}
\usepackage[margin=1in]{geometry}
\usepackage{color}
\usepackage{fancyvrb}
\newcommand{\VerbBar}{|}
\newcommand{\VERB}{\Verb[commandchars=\\\{\}]}
\DefineVerbatimEnvironment{Highlighting}{Verbatim}{commandchars=\\\{\}}
% Add ',fontsize=\small' for more characters per line
\usepackage{framed}
\definecolor{shadecolor}{RGB}{248,248,248}
\newenvironment{Shaded}{\begin{snugshade}}{\end{snugshade}}
\newcommand{\AlertTok}[1]{\textcolor[rgb]{0.94,0.16,0.16}{#1}}
\newcommand{\AnnotationTok}[1]{\textcolor[rgb]{0.56,0.35,0.01}{\textbf{\textit{#1}}}}
\newcommand{\AttributeTok}[1]{\textcolor[rgb]{0.13,0.29,0.53}{#1}}
\newcommand{\BaseNTok}[1]{\textcolor[rgb]{0.00,0.00,0.81}{#1}}
\newcommand{\BuiltInTok}[1]{#1}
\newcommand{\CharTok}[1]{\textcolor[rgb]{0.31,0.60,0.02}{#1}}
\newcommand{\CommentTok}[1]{\textcolor[rgb]{0.56,0.35,0.01}{\textit{#1}}}
\newcommand{\CommentVarTok}[1]{\textcolor[rgb]{0.56,0.35,0.01}{\textbf{\textit{#1}}}}
\newcommand{\ConstantTok}[1]{\textcolor[rgb]{0.56,0.35,0.01}{#1}}
\newcommand{\ControlFlowTok}[1]{\textcolor[rgb]{0.13,0.29,0.53}{\textbf{#1}}}
\newcommand{\DataTypeTok}[1]{\textcolor[rgb]{0.13,0.29,0.53}{#1}}
\newcommand{\DecValTok}[1]{\textcolor[rgb]{0.00,0.00,0.81}{#1}}
\newcommand{\DocumentationTok}[1]{\textcolor[rgb]{0.56,0.35,0.01}{\textbf{\textit{#1}}}}
\newcommand{\ErrorTok}[1]{\textcolor[rgb]{0.64,0.00,0.00}{\textbf{#1}}}
\newcommand{\ExtensionTok}[1]{#1}
\newcommand{\FloatTok}[1]{\textcolor[rgb]{0.00,0.00,0.81}{#1}}
\newcommand{\FunctionTok}[1]{\textcolor[rgb]{0.13,0.29,0.53}{\textbf{#1}}}
\newcommand{\ImportTok}[1]{#1}
\newcommand{\InformationTok}[1]{\textcolor[rgb]{0.56,0.35,0.01}{\textbf{\textit{#1}}}}
\newcommand{\KeywordTok}[1]{\textcolor[rgb]{0.13,0.29,0.53}{\textbf{#1}}}
\newcommand{\NormalTok}[1]{#1}
\newcommand{\OperatorTok}[1]{\textcolor[rgb]{0.81,0.36,0.00}{\textbf{#1}}}
\newcommand{\OtherTok}[1]{\textcolor[rgb]{0.56,0.35,0.01}{#1}}
\newcommand{\PreprocessorTok}[1]{\textcolor[rgb]{0.56,0.35,0.01}{\textit{#1}}}
\newcommand{\RegionMarkerTok}[1]{#1}
\newcommand{\SpecialCharTok}[1]{\textcolor[rgb]{0.81,0.36,0.00}{\textbf{#1}}}
\newcommand{\SpecialStringTok}[1]{\textcolor[rgb]{0.31,0.60,0.02}{#1}}
\newcommand{\StringTok}[1]{\textcolor[rgb]{0.31,0.60,0.02}{#1}}
\newcommand{\VariableTok}[1]{\textcolor[rgb]{0.00,0.00,0.00}{#1}}
\newcommand{\VerbatimStringTok}[1]{\textcolor[rgb]{0.31,0.60,0.02}{#1}}
\newcommand{\WarningTok}[1]{\textcolor[rgb]{0.56,0.35,0.01}{\textbf{\textit{#1}}}}
\usepackage{graphicx}
\makeatletter
\def\maxwidth{\ifdim\Gin@nat@width>\linewidth\linewidth\else\Gin@nat@width\fi}
\def\maxheight{\ifdim\Gin@nat@height>\textheight\textheight\else\Gin@nat@height\fi}
\makeatother
% Scale images if necessary, so that they will not overflow the page
% margins by default, and it is still possible to overwrite the defaults
% using explicit options in \includegraphics[width, height, ...]{}
\setkeys{Gin}{width=\maxwidth,height=\maxheight,keepaspectratio}
% Set default figure placement to htbp
\makeatletter
\def\fps@figure{htbp}
\makeatother
\setlength{\emergencystretch}{3em} % prevent overfull lines
\providecommand{\tightlist}{%
  \setlength{\itemsep}{0pt}\setlength{\parskip}{0pt}}
\setcounter{secnumdepth}{-\maxdimen} % remove section numbering
\ifLuaTeX
  \usepackage{selnolig}  % disable illegal ligatures
\fi
\IfFileExists{bookmark.sty}{\usepackage{bookmark}}{\usepackage{hyperref}}
\IfFileExists{xurl.sty}{\usepackage{xurl}}{} % add URL line breaks if available
\urlstyle{same}
\hypersetup{
  pdftitle={Social Network Analysis über die subreddit Veranstaltung /r/Place2023},
  pdfauthor={Mehmet Karaca},
  hidelinks,
  pdfcreator={LaTeX via pandoc}}

\title{Social Network Analysis über die subreddit Veranstaltung
/r/Place2023}
\author{Mehmet Karaca}
\date{27.10.2023}

\begin{document}
\maketitle

{
\setcounter{tocdepth}{2}
\tableofcontents
}

\section{Einleitung}\label{sec-einleitung}

Auf der Social Media ``Reddit'' gibt es sogenannte Subreddits, welche
bestimmte Foren zu bestimmten Themen sind. Eines dieser Subreddits ist
das Subreddit ``Place''. Die Abkürzung /r/ steht als Abkürzung für ein
Subreddit. Dieses bestimmte subreddit ist ein besonderes, da es nur für
einen bestimmten Zeitraum existiert und jährlich stattfindet. Das erste
Mal fand das Event 2017 statt und war als Aprilscherz gedacht. Am 1.
April 2022 wurde das Event wiederholt und fand zum zweiten Mal statt.
Die letzte Veranstaltung fand vom 20.07.2023 bis zum 25.07.2023 statt.

Des Weiteren gibt es eine weitere Besonderheit. In der Regel ist es in
subreddits nur möglich Beträge in Form von Bildern, Videos oder Texten
zu posten. In diesem Subreddit ist es möglich Pixel auf einer Leinwand
zu platzieren. Dabei können die Nutzer aus einer Palette an Farben
auswählen. Allerdings können die Nutzer die Pixel nur nach Ablauf einer
bestimmten Zeit setzen. Diese Zeit beträgt zwischen 5 und 20 Minuten.
Dadurch ist schwierig als einzelner Nutzer ein Bild zu erstellen, da
andere Nutzer die Pixel überschreiben können. Es ist Teamwork gefragt,
um ein Bild zu erstellen. Dieses Subreddit ist sozusagen soziales
Experiment, welches die Nutzer dazu anregen soll, gemeinsam etwas zu
erstellen. Die Nutzer können sich absprechen und gemeinsam ein Bild
erstellen. Dieses Bild kann ein Logo, eine Flagge oder ein Muster sein.

Für diese Arbeit wurde der Datensatz der letzten Veranstaltung aus dem
Jahr 2023 verwendet. Die Daten wurden von der Webseite
\url{https://www.reddit.com/r/place/comments/15bjm5o/rplace_2023_data/}
heruntergeladen.

Die Motivation für die Arbeit ist, dass es interessant ist zu
untersuchen, wie die Nutzer sich verhalten und wie sie sich absprechen.
Außerdem ist es interessant zu untersuchen, wie die Beziehungen zwischen
den Nutzern sind und wie die Nutzer sich verhalten.

\subsection{Forschungsfragen:}\label{forschungsfragen}

\begin{itemize}
\tightlist
\item
  Sind Muster zu erkennen und in welchen Bereichen der Leinwand sind
  diese Muster zu erkennen?
\item
  Welche Bereiche der Leinwand wurden besonders oft gefüllt?
\item
  Welche Nutzer waren am aktivsten und welche Farben haben sie genutzt?
\item
  Welche Bereiche haben die aktivsten Nutzer besonders oft gefüllt?
\item
  Gibt es generell Beziehungen zwischen den Nutzern?
\end{itemize}

\subsection{Verwendete Librarys}\label{verwendete-librarys}

\begin{Shaded}
\begin{Highlighting}[]
\CommentTok{\#Installieren der benötigten Packages}

\CommentTok{\#install.packages("tidyverse")}
\CommentTok{\#install.packages("ggraph")}
\CommentTok{\#install.packages("ggplot2")}

\CommentTok{\#Laden der benötigten Packages}
\CommentTok{\#library(tidyverse)}
\CommentTok{\#library(ggraph)}
\CommentTok{\#library(ggplot2)}
\end{Highlighting}
\end{Shaded}

\subsection{Datenaufbereitung}\label{datenaufbereitung}

Wie bereits erwähnt, wurde der Datensatz von der Webseite auf Reddit
bereitgestellt. Die Daten liegen als .csv Datei vor und sind sehr groß.
Die Verarbeitung der Daten ist sehr aufwendig. Aufgrund der Größe wurde
der Datensatz auch nicht in Github oder in Moodle hochgeladen. Daher
wurde ein Kompromiss nach Absprache mit der Dozentin getroffen. Es wird
ein Ausschnitt des Datensatzes verwendet, um die Größe des Datensatzes
zu reduzieren. Darüber hinaus wurde der Datensatz vorverarbeitet, um die
Daten zu vereinfachen und die Verarbeitung zu beschleunigen. Der
verarbeitete Datensatz ist in der Abgabe beigefügt, sodass die
Durchführung des RMarkdowns möglich ist. Die Vorverarbeitung wurde mit
Python durchgeführt. Hierfür wurde ein Skript geschrieben, welches die
Daten vorverarbeitet. Das Skript ist in der Abgabe beigefügt. Die
Vorverarbeitung wurde mit Python durchgeführt, da die Verarbeitung mit R
sehr lange gedauert hat. Darüber hinaus wurde die Library Pandas
verwendet, welche sehr gut für die Verarbeitung von Daten geeignet ist.

Die Vorgehensweise der Vorverarbeitung ist wie folgt:

\begin{enumerate}
\def\labelenumi{\arabic{enumi}.}
\item
  Es wurde eine Liste erstellt, welche die betroffenen Dateien enthält.
\item
  Mithilfe einer Schleife wurde jede Datei aus der Liste geöffnet und in
  ein Dataframe umgewandelt.
\item
  Jedes der Dataframes wurde mit der Funktion standardize\_data
  vorverarbeitet.
\item
  In der Funktion wurde die Spalte timestamp angepasst, sodass es die
  Uhrzeit als int Wert enthält.
\item
  Die Datensätze wurden gefiltert, sodass nur die Daten zwischen
  00:00:00 und 15:59:59 enthalten sind.
\item
  Die Spalte pixel\_color (Hexadezimalzahl) wurde angepasst, sodass es
  die Farben als int Wert enthält.
\item
  Fehlerwerte in den Spalten x und y wurden entfernt.
\item
  Die Spalten x und y wurden angepasst, sodass die Koordinaten als int
  Wert enthalten.
\item
  Die Werte in der Spalte user wurden pseudonymisiert, sodass die Nutzer
  anonym bleiben.
\item
  Der verarbeitete Datensatz wird zurückgegeben und alle Dataframes
  werden in einem zusammenhängende Dataframe gespeichert.
\item
  Die Informationen zu dem neuen Dataframe werden ausgegeben
\item
  Der fertige Dataframe wird als neue .csv Datei als
  2023\_place\_canvas\_20072023.csv gespeichert.
\end{enumerate}

\section{Datenanalyse und
Auswertung}\label{sec-datenanalyse-und-auswertung}

Dieser Datensatz wird nun in R geladen und für die weitere Analyse
verwendet. Die Daten werden in einem Dataframe gespeichert und
anschließend werden die Daten untersucht. Es werden erste Tests
durchgeführt, um einen Überblick über die Daten zu bekommen. Hierzu
werden libraries wie tidyverse, igraph, ggplot2 und dplyr verwendet.
Anschließend werden die Daten visualisiert, um weitere Erkenntnisse zu
gewinnen.

\subsection{Erster Überblick über den
Datensatz}\label{erster-uxfcberblick-uxfcber-den-datensatz}

\begin{Shaded}
\begin{Highlighting}[]
\CommentTok{\#Laden der Daten von  /Users/karaca/src/Social\_Network\_Analysis/Datensatz/}
                        \CommentTok{\#2023\_place\_canvas\_20072023.csv}
\NormalTok{data }\OtherTok{\textless{}{-}} \FunctionTok{read.csv}\NormalTok{(}\StringTok{"Datensatz/2023\_place\_canvas\_20072023.csv"}\NormalTok{, }\AttributeTok{sep =} \StringTok{","}\NormalTok{)}

\CommentTok{\#csv Datei in ein Dataframe umwandeln}
\NormalTok{data }\OtherTok{\textless{}{-}} \FunctionTok{as.data.frame}\NormalTok{(data)}

\CommentTok{\#Anzeigen der ersten 6 Zeilen zum testen}
\CommentTok{\#head(data)}

\CommentTok{\#Anzeigen der Struktur der Daten}
\CommentTok{\#str(data)}
\end{Highlighting}
\end{Shaded}

\begin{Shaded}
\begin{Highlighting}[]
\CommentTok{\#install.packages("igraph")}
\FunctionTok{library}\NormalTok{(igraph)}
\end{Highlighting}
\end{Shaded}

\begin{verbatim}
## 
## Attaching package: 'igraph'
\end{verbatim}

\begin{verbatim}
## The following objects are masked from 'package:stats':
## 
##     decompose, spectrum
\end{verbatim}

\begin{verbatim}
## The following object is masked from 'package:base':
## 
##     union
\end{verbatim}

\begin{Shaded}
\begin{Highlighting}[]
\CommentTok{\#install.packages("tidyverse")}
\FunctionTok{library}\NormalTok{(tidyverse)}
\end{Highlighting}
\end{Shaded}

\begin{verbatim}
## -- Attaching core tidyverse packages ------------------------ tidyverse 2.0.0 --
## v dplyr     1.1.3     v readr     2.1.4
## v forcats   1.0.0     v stringr   1.5.0
## v ggplot2   3.4.4     v tibble    3.2.1
## v lubridate 1.9.3     v tidyr     1.3.0
## v purrr     1.0.2
\end{verbatim}

\begin{verbatim}
## -- Conflicts ------------------------------------------ tidyverse_conflicts() --
## x lubridate::%--%()      masks igraph::%--%()
## x dplyr::as_data_frame() masks tibble::as_data_frame(), igraph::as_data_frame()
## x purrr::compose()       masks igraph::compose()
## x tidyr::crossing()      masks igraph::crossing()
## x dplyr::filter()        masks stats::filter()
## x dplyr::lag()           masks stats::lag()
## x purrr::simplify()      masks igraph::simplify()
## i Use the conflicted package (<http://conflicted.r-lib.org/>) to force all conflicts to become errors
\end{verbatim}

\begin{Shaded}
\begin{Highlighting}[]
\CommentTok{\#install.packages(ggplot2)}
\CommentTok{\#library(ggplot2)}

\CommentTok{\#install.packages("dplyr")}
\CommentTok{\#library(dplyr)}
\end{Highlighting}
\end{Shaded}

Da der Datensatz geladen worden ist, führen wir nun erste Tests durch,
um einen Überblick über die Daten zu bekommen.

\begin{Shaded}
\begin{Highlighting}[]
\CommentTok{\#Zuerst lassen wir uns den höchsten und niedrigsten Wert von timestamp anzeigen}
\CommentTok{\#,um einen Überblick über die Zeit zu bekommen.}
\FunctionTok{max}\NormalTok{(data}\SpecialCharTok{$}\NormalTok{timestamp)}
\end{Highlighting}
\end{Shaded}

\begin{verbatim}
## [1] 155959
\end{verbatim}

\begin{Shaded}
\begin{Highlighting}[]
\FunctionTok{min}\NormalTok{(data}\SpecialCharTok{$}\NormalTok{timestamp)}
\end{Highlighting}
\end{Shaded}

\begin{verbatim}
## [1] 130026
\end{verbatim}

\begin{Shaded}
\begin{Highlighting}[]
\CommentTok{\#Als Nächstes schauen wir uns an, wie viele einzigartige Werte es in der Spalte user gibt.}
\CommentTok{\#Dies gibt uns die Anzahl an Usern an, da manche User mehrere Einträge haben.}
\FunctionTok{length}\NormalTok{(}\FunctionTok{unique}\NormalTok{(data}\SpecialCharTok{$}\NormalTok{user))}
\end{Highlighting}
\end{Shaded}

\begin{verbatim}
## [1] 739675
\end{verbatim}

\begin{Shaded}
\begin{Highlighting}[]
\CommentTok{\#Der Datensatz wurde im Zeitraum von 13:00:26 bis 15:59:59 aufgenommen.}
\CommentTok{\#In diesem Zeitraum haben 739675 User insgesamt 2411941 Pixel gesetzt.}
\CommentTok{\#Die Pixel Color wurde als int Wert gespeichert, durch die Vorverarbeitung}
\CommentTok{\#des Datensatzes. Die Zahlen haben einen bestimmten Schlüssel für die Farbe.}
\CommentTok{\#Tabelle der Farben}
\CommentTok{\#rot = \#ff4500 = 1}
\CommentTok{\#orange = \#ffa800 = 2}
\CommentTok{\#gelb = \#ffd635  = 3}
\CommentTok{\#grün = \#00a368 = 4}
\CommentTok{\#blau = \#3690ea = 5}
\CommentTok{\#lila = \#b44ac0 = 6}
\CommentTok{\#schwarz = \#000000 = 7}
\CommentTok{\#weiß = \#ffffff = 8}
\end{Highlighting}
\end{Shaded}

\begin{Shaded}
\begin{Highlighting}[]
\CommentTok{\#Diese Farben nutzen wir um einene Farbzuordnungstabelle zu erstellen}
\NormalTok{farbzuordnung }\OtherTok{\textless{}{-}} \FunctionTok{c}\NormalTok{(}\StringTok{"\#ff4500"}\NormalTok{, }\StringTok{"\#ffa800"}\NormalTok{, }\StringTok{"\#ffd635"}\NormalTok{, }\StringTok{"\#00a368"}\NormalTok{,}
                \StringTok{"\#3690ea"}\NormalTok{, }\StringTok{"\#b44ac0"}\NormalTok{, }\StringTok{"\#000000"}\NormalTok{, }\StringTok{"\#ffffff"}\NormalTok{)}
\end{Highlighting}
\end{Shaded}

Als Nächstes zeigen wir uns die Maximal- und Minimalwerte von x und y
anzeigen, um einen Eindruck des Koordinatensystems zu bekommen.

\begin{Shaded}
\begin{Highlighting}[]
\FunctionTok{max}\NormalTok{(data}\SpecialCharTok{$}\NormalTok{x)}
\end{Highlighting}
\end{Shaded}

\begin{verbatim}
## [1] 499
\end{verbatim}

\begin{Shaded}
\begin{Highlighting}[]
\FunctionTok{min}\NormalTok{(data}\SpecialCharTok{$}\NormalTok{x)}
\end{Highlighting}
\end{Shaded}

\begin{verbatim}
## [1] -500
\end{verbatim}

\begin{Shaded}
\begin{Highlighting}[]
\FunctionTok{max}\NormalTok{(data}\SpecialCharTok{$}\NormalTok{y)}
\end{Highlighting}
\end{Shaded}

\begin{verbatim}
## [1] 499
\end{verbatim}

\begin{Shaded}
\begin{Highlighting}[]
\FunctionTok{min}\NormalTok{(data}\SpecialCharTok{$}\NormalTok{y)}
\end{Highlighting}
\end{Shaded}

\begin{verbatim}
## [1] -500
\end{verbatim}

Das Koordinatensystem erstreckt sich von -500 bis 499 entlang sowohl der
X-Achse als auch der Y-Achse. Wie bereits erwähnt ist die Leinwand
1000x1000 Pixel groß und passt somit zu dem Koordinatensystem. Die
Koordinate 500\textbar500 wurde nicht verwendet, daher ergeben sich als
Max-Werte 499\textbar499. Der Nullpunkt 0\textbar0 befindet sich in der
Mitte des Koordinatensystems.

Als nächsten Schritt wollen wir die Daten mit einem Plot visualisieren,
um einen Eindruck der Daten zu bekommen. Hierfür erstellen wir ein
Subset, welches nur aus den Koordinatenpaaren besteht.

\begin{Shaded}
\begin{Highlighting}[]
\CommentTok{\#Erstelle einen Subset, welcher nur aus den Koordinatenpaaren besteht}
\NormalTok{Koordinaten\_subset\_x\_und\_y }\OtherTok{\textless{}{-}}\NormalTok{ data[,}\FunctionTok{c}\NormalTok{(}\StringTok{"x"}\NormalTok{, }\StringTok{"y"}\NormalTok{)]}

\CommentTok{\#Diesen Subset wollen wir uns anschauen und plotten diesen in einem Scatterplot.}
\CommentTok{\#plot(Koordinaten\_subset\_x\_und\_y$x, Koordinaten\_subset\_x\_und\_y$y, xlab = "X{-}Achse", }
      \CommentTok{\#ylab = "Y{-}Achse")}
\CommentTok{\#Aus Erkenntniss Gründen wurde der Plot auskommentiert, da dieser nur ein schwarzes }
  \CommentTok{\#Feld darstellt.}
\end{Highlighting}
\end{Shaded}

An dem Plot erkennt man, das beinahe alle Koordinaten mit einem Pixel
belegt worden sind. Es gibt nur vereinzelte Ausnahmen, die nicht belegt
worden sind. Aus dem Plot können keine tiefergehenden Erkenntnisse
gezogen werden. Daher soll im nächsten Schritt die Punkte farbig
visualisiert werden, um bestimmte Muster erkennen zu können. Hierzu
erstellen wir einen neuen Subset, welcher aus den Koordinatenpaaren und
dem pixelcolor besteht. Diesen subset wollen wir uns anschauen und
plotten diesen auf einem Scatterplot mit der Farbe des pixel\_color. Die
Farben sind in der Tabelle der Farben oben beschrieben.

\begin{Shaded}
\begin{Highlighting}[]
\CommentTok{\# Erstellen eines neuen Subsets,}
\CommentTok{\#welches aus den Koordinatenpaaren und dem pixelcolor besteht.}
\NormalTok{Koordinaten\_subset\_x\_y\_pixelcolor }\OtherTok{\textless{}{-}} \FunctionTok{data.frame}\NormalTok{(}\AttributeTok{x=}\NormalTok{data}\SpecialCharTok{$}\NormalTok{x, }\AttributeTok{y=}\NormalTok{data}\SpecialCharTok{$}\NormalTok{y, }
  \AttributeTok{pixel\_color=}\NormalTok{data}\SpecialCharTok{$}\NormalTok{pixel\_color)}

\CommentTok{\#Diesen Subset wollen wir uns anschauen und plotten diesen auf einem Scatterplot }
\CommentTok{\#mit der Farbe des pixelcolors}
\FunctionTok{plot}\NormalTok{(Koordinaten\_subset\_x\_y\_pixelcolor}\SpecialCharTok{$}\NormalTok{x, Koordinaten\_subset\_x\_y\_pixelcolor}\SpecialCharTok{$}\NormalTok{y,}
      \AttributeTok{col =}\NormalTok{ farbzuordnung[Koordinaten\_subset\_x\_y\_pixelcolor}\SpecialCharTok{$}\NormalTok{pixel\_color] ,}\AttributeTok{pch=}\DecValTok{15}\NormalTok{,}
      \AttributeTok{xlab =} \StringTok{"X{-}Achse"}\NormalTok{, }\AttributeTok{ylab =} \StringTok{"Y{-}Achse"}\NormalTok{)}
\end{Highlighting}
\end{Shaded}

\includegraphics{SNA-Abgabe_files/figure-latex/unnamed-chunk-9-1.pdf}

Durch diese Darstellung ist nun zu erkennen, dass es bestimmte
Farbmuster gibt. Beispielsweise sind erste Flaggen zu erkennen wie die
Französische (links), die Deutsche (links unten), türkische, indische,
italienische. Allerdings haben wir ein Problem bei dieser
Visualisierung, da viele Punkte doppelt vorkommen. Dies liegt daran,
dass auf einem Pixel mehrere Pixel gesetzt worden sind. Deswegen möchten
wir nur die letzten Pixel setzen lassen. Hierfür erstellen wir einen
neuen Subset. In diesem subset gibt es keine doppelten Koordinatenpaare.

\begin{Shaded}
\begin{Highlighting}[]
\CommentTok{\# Entfernen von doppelten Koordinatenpaaren, um nur die letzten Pixel zu behalten}
\NormalTok{Koordinaten\_subset\_eindeutig }\OtherTok{\textless{}{-}} 
\NormalTok{Koordinaten\_subset\_x\_y\_pixelcolor[}\SpecialCharTok{!}\FunctionTok{duplicated}\NormalTok{(Koordinaten\_subset\_x\_y\_pixelcolor[,}
  \FunctionTok{c}\NormalTok{(}\StringTok{"x"}\NormalTok{, }\StringTok{"y"}\NormalTok{)], }\AttributeTok{fromLast =} \ConstantTok{TRUE}\NormalTok{), ]}

\CommentTok{\# Plot der eindeutigen Koordinatenpaare mit der Farbe des letzten pixel\_color}
\FunctionTok{plot}\NormalTok{(Koordinaten\_subset\_eindeutig}\SpecialCharTok{$}\NormalTok{x, Koordinaten\_subset\_eindeutig}\SpecialCharTok{$}\NormalTok{y, }
\AttributeTok{col =}\NormalTok{ farbzuordnung[Koordinaten\_subset\_eindeutig}\SpecialCharTok{$}\NormalTok{pixel\_color], }\AttributeTok{pch =} \DecValTok{15}\NormalTok{,}
    \AttributeTok{xlab =} \StringTok{"X{-}Achse"}\NormalTok{, }\AttributeTok{ylab =} \StringTok{"Y{-}Achse"}\NormalTok{)}
\end{Highlighting}
\end{Shaded}

\includegraphics{SNA-Abgabe_files/figure-latex/unnamed-chunk-10-1.pdf}

Die Genauigkeit und Präzision der Grafik sind nun deutlich besser. Es
sind keine doppelten Koordinatenpaare mehr vorhanden. Allerdings ist
dies noch nicht perfekt. Beim Ausprobieren mit ggplot2 hat sich gezeigt,
dass bei der Verwendung von ggplot2 die Grafik noch präziser wird. Daher
wird im nächsten Schritt die Grafik mit ggplot2 geplottet.

\begin{Shaded}
\begin{Highlighting}[]
\FunctionTok{library}\NormalTok{ (ggplot2)}
\CommentTok{\#Plotten der Grafik in ggplot}
\FunctionTok{ggplot}\NormalTok{(Koordinaten\_subset\_eindeutig, }\FunctionTok{aes}\NormalTok{(x, y, }\AttributeTok{col =}\NormalTok{ farbzuordnung[pixel\_color])) }\SpecialCharTok{+}
  \FunctionTok{geom\_point}\NormalTok{(}\AttributeTok{shape =} \DecValTok{15}\NormalTok{, }\AttributeTok{size =} \FloatTok{0.4}\NormalTok{) }\SpecialCharTok{+}
  \FunctionTok{scale\_color\_identity}\NormalTok{() }\SpecialCharTok{+}
  \FunctionTok{theme\_minimal}\NormalTok{() }\SpecialCharTok{+}
  \FunctionTok{labs}\NormalTok{(}\AttributeTok{title =} \StringTok{"Koordinatenpaare mit Farbe"}\NormalTok{, }\AttributeTok{x =} \StringTok{"X{-}Achse"}\NormalTok{, }\AttributeTok{y =} \StringTok{"Y{-}Achse"}\NormalTok{, }\AttributeTok{col =} \StringTok{"Farbe"}\NormalTok{)}
\end{Highlighting}
\end{Shaded}

\includegraphics{SNA-Abgabe_files/figure-latex/unnamed-chunk-11-1.pdf}

Diese Darstellung ist die präziseste Darstellung, die wir erstellen
konnten. Es ist zu erkennen, dass die Flaggen deutlicher zu erkennen
sind. Außerdem sind Sätze, Wörter und Logos zu erkennen. Aber auch
Andeutung von Bildern wie z.B. ein blauer Elefant und ein ``Pikachu''
mit Sonnenbrille (Ein Pokemon aus dem gleichnamigen Spiel) sind der
unteren rechten Ecke ist zu erkennen.

\subsection{Untersuchung der Heatmap der
Koordinatenpaare}\label{untersuchung-der-heatmap-der-koordinatenpaare}

Im folgenden Schritt soll die Quantität untersucht werden. Hierzu soll
herausgefunden werden, welche Bereche am meisten Pixel gesetzt bekommen
haben. Es soll eine Heatmap erstellt werden, welcher die Bereiche mit
den meisten Färbungen anzeigt. Hierfür erstellen wir einen neuen Subset,
welcher aus den Koordinatenpaaren besteht.

\begin{Shaded}
\begin{Highlighting}[]
\CommentTok{\#Erstellen eines neuen Subsets, welches aus den Koordinatenpaaren besteht}
\NormalTok{Koordinaten\_subset\_x\_y\_heatmap }\OtherTok{\textless{}{-}} \FunctionTok{data.frame}\NormalTok{(}\AttributeTok{x=}\NormalTok{data}\SpecialCharTok{$}\NormalTok{x, }\AttributeTok{y=}\NormalTok{data}\SpecialCharTok{$}\NormalTok{y)}

\CommentTok{\#Füge eine neue Spalte hinzu, welcher die Koordinatenpaare zusammenfasst}
\NormalTok{Koordinaten\_subset\_x\_y\_heatmap }\OtherTok{\textless{}{-}} \FunctionTok{transform}\NormalTok{(Koordinaten\_subset\_x\_y\_heatmap,}
 \AttributeTok{x\_y =} \FunctionTok{paste}\NormalTok{(x, y, }\AttributeTok{sep =} \StringTok{"\_"}\NormalTok{))}

\CommentTok{\#Die doppelten Werte von x\_y werden zusammengefasst und gezählt }
\CommentTok{\#und in der Spalte count gespeichert und anschließend }
\CommentTok{\#nach der Anzahl der Koordinatenpaare absteigend sortiert}
\NormalTok{Koordinaten\_subset\_x\_y\_heatmap }\OtherTok{\textless{}{-}}\NormalTok{ Koordinaten\_subset\_x\_y\_heatmap }\SpecialCharTok{\%\textgreater{}\%} \FunctionTok{group\_by}\NormalTok{(x\_y) }\SpecialCharTok{\%\textgreater{}\%} 
\FunctionTok{summarise}\NormalTok{(}\AttributeTok{count =} \FunctionTok{n}\NormalTok{()) }\SpecialCharTok{\%\textgreater{}\%} \FunctionTok{arrange}\NormalTok{(}\FunctionTok{desc}\NormalTok{(count))}

\CommentTok{\#Nun sollen doppelte Koordinatenpaare x\_y entfernt werden, um nur die eindeutigen }
\CommentTok{\#Koordinatenpaare zu behalten}
\NormalTok{Koordinaten\_subset\_x\_y\_heatmap }\OtherTok{\textless{}{-}}\NormalTok{ Koordinaten\_subset\_x\_y\_heatmap[}
  \SpecialCharTok{!}\FunctionTok{duplicated}\NormalTok{(Koordinaten\_subset\_x\_y\_heatmap}\SpecialCharTok{$}\NormalTok{x\_y), ]}

\CommentTok{\#Zeige die ersten 5 Zeilen des Subsets an zum testen}
\CommentTok{\#head(Koordinaten\_subset\_x\_y\_heatmap)}

\CommentTok{\# Zeige den größten und kleinsten Wert von count an}
\CommentTok{\#max(Koordinaten\_subset\_x\_y\_heatmap$count)}
\CommentTok{\#min(Koordinaten\_subset\_x\_y\_heatmap$count)}

\CommentTok{\# Erstelle die Dichtefunktion}
\NormalTok{density\_function }\OtherTok{\textless{}{-}} \FunctionTok{ecdf}\NormalTok{(Koordinaten\_subset\_x\_y\_heatmap}\SpecialCharTok{$}\NormalTok{count)}

\CommentTok{\# Plotte die Dichtefunktion}
\FunctionTok{plot}\NormalTok{(density\_function, }\AttributeTok{xlim=}\FunctionTok{c}\NormalTok{(}\DecValTok{0}\NormalTok{, }\DecValTok{2000}\NormalTok{), }\AttributeTok{ylim=}\FunctionTok{c}\NormalTok{(}\DecValTok{0}\NormalTok{, }\DecValTok{1}\NormalTok{),}
     \AttributeTok{xlab=}\StringTok{"Anzahl der Koordinatenpaare"}\NormalTok{, }
     \AttributeTok{ylab=}\StringTok{"Kumulierte Wahrscheinlichkeit"}\NormalTok{, }
     \AttributeTok{main=}\StringTok{"Kumulierte Dichtefunktion der Koordinatenpaare"}\NormalTok{)}
\end{Highlighting}
\end{Shaded}

\includegraphics{SNA-Abgabe_files/figure-latex/unnamed-chunk-12-1.pdf}

Die kumulierte Dichtefunktion der Koordinatenpaare wurde benötigt, da
vorher nicht ersichtlich war, wie die Verteilung der Koordinatenpaare
ist. Ohne diese Erkenntnis war es durchaus schwierig eine passende
Farbpalette zu erstellen, da die Verteilung sehr ungleichmäßig ist.
Diese ungleichmäßige Verteilung mit einer gleichmäßigen Farbpalette
darzustellen, würde zu einer falschen Darstellung führen. Beim
Ausprobieren mit einer gleichmäßigen Farbpalette war die Visualisierung
nicht aussagekräftig. Diese Erkenntnis hat sich im Verlauf der
Untersuchung herausgestellt, weshalb Anpassungen vorgenommen wurden. Nun
ist es möglich eine aussagekräftige Heatmap zu erstellen.

\begin{Shaded}
\begin{Highlighting}[]
\CommentTok{\#Erstellen einer Tabelle, welche die Häufigkeit der Koordinatenpaare zählt}
\NormalTok{Koordinaten\_subset\_x\_y\_heatmap\_tabelle }\OtherTok{\textless{}{-}} \FunctionTok{data.frame}\NormalTok{(}\AttributeTok{x=}\NormalTok{data}\SpecialCharTok{$}\NormalTok{x, }\AttributeTok{y=}\NormalTok{data}\SpecialCharTok{$}\NormalTok{y)}

\CommentTok{\# Zähle die Häufigkeit der Koordinatenpaare und erstelle eine neue Tabelle}
\NormalTok{häufigkeit\_tabelle }\OtherTok{\textless{}{-}} \FunctionTok{as.data.frame}\NormalTok{(}\FunctionTok{table}\NormalTok{(Koordinaten\_subset\_x\_y\_heatmap\_tabelle))}

\CommentTok{\# Benenne die Spalten um}
\FunctionTok{colnames}\NormalTok{(häufigkeit\_tabelle) }\OtherTok{\textless{}{-}} \FunctionTok{c}\NormalTok{(}\StringTok{"X"}\NormalTok{, }\StringTok{"Y"}\NormalTok{, }\StringTok{"Anzahl"}\NormalTok{)}

\CommentTok{\#Füge eine neue Spalte hinzu, welche den Koordinatenpaare Farben zuweist,}
\CommentTok{\#je nach Anzahl der Koordinatenpaare}
\CommentTok{\#Die Farbbereiche werden durch eine passende Auswahl definiert.}
\NormalTok{häufigkeit\_tabelle}\SpecialCharTok{$}\NormalTok{Farbe }\OtherTok{\textless{}{-}} \FunctionTok{ifelse}\NormalTok{(häufigkeit\_tabelle}\SpecialCharTok{$}\NormalTok{Anzahl }\SpecialCharTok{==} \DecValTok{0}\NormalTok{, }\StringTok{"white"}\NormalTok{,}
\FunctionTok{ifelse}\NormalTok{(häufigkeit\_tabelle}\SpecialCharTok{$}\NormalTok{Anzahl }\SpecialCharTok{\textless{}} \DecValTok{10}\NormalTok{, }\StringTok{"purple"}\NormalTok{,}
\FunctionTok{ifelse}\NormalTok{(häufigkeit\_tabelle}\SpecialCharTok{$}\NormalTok{Anzahl }\SpecialCharTok{\textless{}} \DecValTok{50}\NormalTok{, }\StringTok{"blue"}\NormalTok{,}
\FunctionTok{ifelse}\NormalTok{(häufigkeit\_tabelle}\SpecialCharTok{$}\NormalTok{Anzahl }\SpecialCharTok{\textless{}} \DecValTok{350}\NormalTok{, }\StringTok{"orange"}\NormalTok{, }\StringTok{"red"}\NormalTok{))))}

\CommentTok{\# Zeige einen Ausschnitt der Tabelle an zum testen}
\CommentTok{\#print(head(häufigkeit\_tabelle))}

\CommentTok{\#Anzahl der Zeilen der Tabelle}
\CommentTok{\#nrow(häufigkeit\_tabelle)}

\CommentTok{\#Legende erstellen für die Heatmap}
\NormalTok{Legenden\_Daten }\OtherTok{\textless{}{-}} \FunctionTok{data.frame}\NormalTok{(}\AttributeTok{Farbe =} \FunctionTok{unique}\NormalTok{(häufigkeit\_tabelle}\SpecialCharTok{$}\NormalTok{Farbe),}
                      \AttributeTok{Bedeutung =} \FunctionTok{c}\NormalTok{(}\StringTok{"\textgreater{}350"}\NormalTok{, }\StringTok{"\textless{}350"}\NormalTok{, }\StringTok{"\textless{}50"}\NormalTok{, }\StringTok{"\textless{}10"}\NormalTok{, }\StringTok{"0"}\NormalTok{))}
\end{Highlighting}
\end{Shaded}

\begin{Shaded}
\begin{Highlighting}[]
\CommentTok{\#Nun erstellen wir einen Plot, welcher die X{-}Achse und Y{-}Achse}
\CommentTok{\#der Koordinatenpaare mit ihrer Farbe darstellt.}
\NormalTok{Koordinaten\_subset\_x\_y\_heatmap\_tabelle\_grafik }\OtherTok{\textless{}{-}} \FunctionTok{ggplot}\NormalTok{(}
\NormalTok{  häufigkeit\_tabelle,}\FunctionTok{aes}\NormalTok{(X, Y, }\AttributeTok{col =}\NormalTok{ Farbe)) }\SpecialCharTok{+} 
  \FunctionTok{geom\_point}\NormalTok{(}\AttributeTok{shape =} \DecValTok{15}\NormalTok{, }\AttributeTok{size =} \FloatTok{0.5}\NormalTok{) }\SpecialCharTok{+} \FunctionTok{scale\_color\_manual}\NormalTok{(}\AttributeTok{values =} \FunctionTok{as.character}\NormalTok{(häufigkeit\_tabelle}\SpecialCharTok{$}\NormalTok{Farbe),}
  \AttributeTok{labels =}\NormalTok{ Legenden\_Daten}\SpecialCharTok{$}\NormalTok{Bedeutung) }\SpecialCharTok{+} \FunctionTok{theme\_minimal}\NormalTok{() }\SpecialCharTok{+}
  \FunctionTok{theme}\NormalTok{(}\AttributeTok{axis.title =} \FunctionTok{element\_blank}\NormalTok{(), }\AttributeTok{axis.text =} \FunctionTok{element\_blank}\NormalTok{(),}
  \AttributeTok{axis.line =} \FunctionTok{element\_blank}\NormalTok{(), }\AttributeTok{axis.ticks =} \FunctionTok{element\_blank}\NormalTok{()) }\SpecialCharTok{+}
  \FunctionTok{labs}\NormalTok{(}\AttributeTok{title =} \StringTok{"Heatmap Anzahl Platzierungen"}\NormalTok{, }\AttributeTok{col =} \StringTok{"Farbe"}\NormalTok{) }\SpecialCharTok{+}
  \FunctionTok{guides}\NormalTok{(}\AttributeTok{col =} \FunctionTok{guide\_legend}\NormalTok{(}\AttributeTok{title =} \StringTok{"Legende"}\NormalTok{))}


\CommentTok{\#Anzeigen der Grafik}
\NormalTok{Koordinaten\_subset\_x\_y\_heatmap\_tabelle\_grafik}
\end{Highlighting}
\end{Shaded}

\includegraphics{SNA-Abgabe_files/figure-latex/unnamed-chunk-14-1.pdf}

Durch die Heatmap erkennt man sehr schön, welche Bereiche am meisten
Pixel gesetzt bekommen haben. Die roten Bereiche wurden relativ oft
gesetzt. Es gibt wenige Bereiche die rote sind. Es gibt Anhäufungen in
den Ecken des Koordinatensystems mit Ausnahme der rechten unteren Ecke.
Außerdem gibt es in der Mitte des Koordinatensystems kleinere
Anhäufungen von Pixeln. In diesen roten Bereichen kann es durchaus sein,
dass die Teilnehmer an der Veranstaltungen einander überboten haben. Die
Annahme, dass die roten Bereiche entstanden sind, um bestimmte Bilder zu
erzeugen und gemeinsam zu erstellen, kann widerlegt werden durch die
Betrachtung der blauen Bereiche der Heatmap. In den blauen Bereichen
sind zusammenhängende Bereiche und Muster zu erkennen. Diese Bereiche
sind relativ groß und wurde daher nicht versucht zu zerstören oder zu
überschreiben. Es sind vereinzelt auch Buchstaben erkenntlich, welche in
den blauen Bereichen entstanden sind. Besonders auffällig ist der
vertikale blaue Streifen auf der linken Seite der Heatmap. Zudem ist ein
weiterer horizontaler blauer Streifen in der unteren Seite der Heatmap
zu erkennen.

\subsection{Untersuchung der Hauptakteure (Stehen die Hauptakteure in
Beziehung zu
einander?)}\label{untersuchung-der-hauptakteure-stehen-die-hauptakteure-in-beziehung-zu-einander}

Da nun ein grober Überblick der gesamten Daten und des
Koordinatensystems vorhanden ist, soll nun untersucht werden, welche
User die Hauptakteure sind. Hierfür erstellen wir einen neuen Subset,
welcher aus den Koordinatenpaaren, dem pixelcolor, dem timestamp und dem
user besteht. Diesen Subset filtern wir nach den Top 10 Usern, welche
die meisten Pixel gesetzt haben. Dieser Subset ist die Grundlage für die
weiteren Untersuchungen zu den Pixel\_Farben und den Koordinatenpaaren.

\begin{Shaded}
\begin{Highlighting}[]
\CommentTok{\#Erstellen eines neuen Subsets, welches aus den Koordinatenpaaren, dem pixelcolor,}
\CommentTok{\#dem timestamp und dem user besteht}
\NormalTok{Top10User }\OtherTok{\textless{}{-}} \FunctionTok{data.frame}\NormalTok{(}\AttributeTok{x=}\NormalTok{data}\SpecialCharTok{$}\NormalTok{x, }\AttributeTok{y=}\NormalTok{data}\SpecialCharTok{$}\NormalTok{y, }\AttributeTok{pixel\_color=}\NormalTok{data}\SpecialCharTok{$}\NormalTok{pixel\_color,}
 \AttributeTok{timestamp=}\NormalTok{data}\SpecialCharTok{$}\NormalTok{timestamp, }\AttributeTok{user=}\NormalTok{data}\SpecialCharTok{$}\NormalTok{user)}

\CommentTok{\# Filtern des Subsets nach den Top 10 Usern,}
\CommentTok{\#welche die meisten Pixel gesetzt haben}
\NormalTok{Top10User }\OtherTok{\textless{}{-}}\NormalTok{ data }\SpecialCharTok{\%\textgreater{}\%} \FunctionTok{group\_by}\NormalTok{(user) }\SpecialCharTok{\%\textgreater{}\%} \FunctionTok{summarise}\NormalTok{(}\AttributeTok{count =} \FunctionTok{n}\NormalTok{()) }\SpecialCharTok{\%\textgreater{}\%}
  \FunctionTok{arrange}\NormalTok{(}\FunctionTok{desc}\NormalTok{(count)) }\SpecialCharTok{\%\textgreater{}\%} \FunctionTok{head}\NormalTok{(}\DecValTok{10}\NormalTok{)}

\CommentTok{\# Filtern des ursprünglichen Dataframes nach den Top 20 Usern}
\NormalTok{Top10User }\OtherTok{\textless{}{-}}\NormalTok{ data }\SpecialCharTok{\%\textgreater{}\%} \FunctionTok{filter}\NormalTok{(user }\SpecialCharTok{\%in\%}\NormalTok{ Top10User}\SpecialCharTok{$}\NormalTok{user)}

\CommentTok{\#Anzeigen der Spalten von Top10User}
\CommentTok{\#colnames(Top10User)}
\CommentTok{\#Top10User}

\CommentTok{\#printe die unique User zum testen}
\CommentTok{\#unique(Top10User$user)}

\CommentTok{\#printe die unique Farben zum testen}
\CommentTok{\#unique(Top10User$pixel\_color)}
\end{Highlighting}
\end{Shaded}

\subsection{Untersuchung der Beziehung der Top10 User im Kontext der
Farben}\label{untersuchung-der-beziehung-der-top10-user-im-kontext-der-farben}

Als nächstes wollen wir untersuchen, welche Farben die Top 10 User
gesetzt haben und wie diese in Verbindung stehen. Hierfür passen wir den
Subset Top10User an, sodass dieser nur noch aus den Spalten user und
pixel\_color besteht.

\begin{Shaded}
\begin{Highlighting}[]
\CommentTok{\#Die doppelten Werte von user und pixel\_color werden zusammengefasst und gezählt}
\CommentTok{\#und in der Spalte count gespeichert und anschließend nach der Anzahl}
\CommentTok{\#absteigend sortiert. Der Dataframe benötigt user und pixel\_color als Vektoren}
\CommentTok{\# und count als Attribut(Gewichtung) der Kanten.}
\NormalTok{Top10UserFarben }\OtherTok{\textless{}{-}}\NormalTok{ Top10User }\SpecialCharTok{\%\textgreater{}\%} \FunctionTok{group\_by}\NormalTok{(user, pixel\_color) }\SpecialCharTok{\%\textgreater{}\%} \FunctionTok{summarise}\NormalTok{(}
  \AttributeTok{count =} \FunctionTok{n}\NormalTok{()) }\SpecialCharTok{\%\textgreater{}\%} \FunctionTok{arrange}\NormalTok{(}\FunctionTok{desc}\NormalTok{(count))}
\end{Highlighting}
\end{Shaded}

\begin{verbatim}
## `summarise()` has grouped output by 'user'. You can override using the
## `.groups` argument.
\end{verbatim}

\begin{Shaded}
\begin{Highlighting}[]
\CommentTok{\#Ausschreiben der Farben in pixel\_color als Wörter statt Zahlen}
\CommentTok{\#für die spätere Verwendung. Der Hintergrund des Plots ist weiß,}
\CommentTok{\#weshalb die Farbe Weiß nicht verwendet werden kann.}
\CommentTok{\#Daher wird die Farbe Pink verwendet.}

\NormalTok{Top10UserFarben}\SpecialCharTok{$}\NormalTok{pixel\_color }\OtherTok{\textless{}{-}} \FunctionTok{ifelse}\NormalTok{(Top10UserFarben}\SpecialCharTok{$}\NormalTok{pixel\_color }\SpecialCharTok{==} \DecValTok{1}\NormalTok{, }\StringTok{"red"}\NormalTok{,}
\FunctionTok{ifelse}\NormalTok{(Top10UserFarben}\SpecialCharTok{$}\NormalTok{pixel\_color }\SpecialCharTok{==} \DecValTok{2}\NormalTok{, }\StringTok{"orange"}\NormalTok{,}
\FunctionTok{ifelse}\NormalTok{(Top10UserFarben}\SpecialCharTok{$}\NormalTok{pixel\_color }\SpecialCharTok{==} \DecValTok{3}\NormalTok{, }\StringTok{"yellow"}\NormalTok{,}
\FunctionTok{ifelse}\NormalTok{(Top10UserFarben}\SpecialCharTok{$}\NormalTok{pixel\_color }\SpecialCharTok{==} \DecValTok{4}\NormalTok{, }\StringTok{"green"}\NormalTok{,}
\FunctionTok{ifelse}\NormalTok{(Top10UserFarben}\SpecialCharTok{$}\NormalTok{pixel\_color }\SpecialCharTok{==} \DecValTok{5}\NormalTok{, }\StringTok{"blue"}\NormalTok{,}
\FunctionTok{ifelse}\NormalTok{(Top10UserFarben}\SpecialCharTok{$}\NormalTok{pixel\_color }\SpecialCharTok{==} \DecValTok{6}\NormalTok{, }\StringTok{"purple"}\NormalTok{,}
\FunctionTok{ifelse}\NormalTok{(Top10UserFarben}\SpecialCharTok{$}\NormalTok{pixel\_color }\SpecialCharTok{==} \DecValTok{7}\NormalTok{, }\StringTok{"black"}\NormalTok{, }\StringTok{"pink"}\NormalTok{)))))))}

\CommentTok{\#Anzeigen von Top10User$pixel\_color zum testen}
\CommentTok{\#Top10User$pixel\_color}

\CommentTok{\#Erstellen von Vektoren für die Knoten und Kanten des Graphen}
\NormalTok{usernamen }\OtherTok{\textless{}{-}} \FunctionTok{c}\NormalTok{(Top10UserFarben}\SpecialCharTok{$}\NormalTok{user)}
\NormalTok{pixel\_color }\OtherTok{\textless{}{-}} \FunctionTok{c}\NormalTok{(Top10UserFarben}\SpecialCharTok{$}\NormalTok{pixel\_color)}
\NormalTok{gewichtung }\OtherTok{\textless{}{-}} \FunctionTok{c}\NormalTok{(Top10UserFarben}\SpecialCharTok{$}\NormalTok{count)}

\CommentTok{\#Anzahl der Einträge in den Vektoren zum testen.}
\CommentTok{\#length(usernamen)}
\CommentTok{\#length(pixel\_color1)}
\CommentTok{\#length(gewichtung)}
\CommentTok{\#usernamen}
\CommentTok{\#gewichtung}
\CommentTok{\#pixel\_color1}

\CommentTok{\#Erstellen einer Matrix aus den Vektoren}
\NormalTok{Top10UserFarben\_Matrix }\OtherTok{\textless{}{-}} \FunctionTok{cbind}\NormalTok{(usernamen, pixel\_color)}

\CommentTok{\#Funktionen zum testen}
\CommentTok{\#Top10User\_namenUndFarben }
\CommentTok{\#class(Top10User\_namenUndFarben)}

\CommentTok{\#Erstellen eines Graphen aus dem Dataframe (Wie in der Vorleusung gezeigt)}
\NormalTok{Top10UserFarben\_Netzwerk }\OtherTok{\textless{}{-}} \FunctionTok{graph\_from\_data\_frame}\NormalTok{(Top10UserFarben\_Matrix, }\AttributeTok{directed =} \ConstantTok{FALSE}\NormalTok{)}

\CommentTok{\#Testen ob der Graph erstellt worden ist}
\CommentTok{\#Top10User\_Netzwerk}

\CommentTok{\#Hinzufügen von Attributen zu den Knoten und Kanten}
\NormalTok{Top10UserFarben\_Netzwerk }\OtherTok{\textless{}{-}} \FunctionTok{set\_edge\_attr}\NormalTok{(Top10UserFarben\_Netzwerk, }
  \StringTok{"gewichtung"}\NormalTok{, }\AttributeTok{value =}\NormalTok{ gewichtung)}

\NormalTok{Top10UserFarben\_Netzwerk }\OtherTok{\textless{}{-}} \FunctionTok{set\_edge\_attr}\NormalTok{(Top10UserFarben\_Netzwerk, }
  \StringTok{"pixel\_color"}\NormalTok{, }\AttributeTok{value =}\NormalTok{ pixel\_color)}

\CommentTok{\#Testen ob die Attribute hinzugefügt worden sind}
\CommentTok{\#Top10User\_Netzwerk\_gewichtet}


\CommentTok{\#Die Knotenfarbe wird hier bestimmt und soll}
\CommentTok{\#sich von den Farben der Kanten unterscheiden.}
\NormalTok{vertex\_colors }\OtherTok{\textless{}{-}} \FunctionTok{rep}\NormalTok{(}\StringTok{"beige"}\NormalTok{, }\FunctionTok{vcount}\NormalTok{(Top10UserFarben\_Netzwerk))}

\CommentTok{\#Plotten des Graphen}
\FunctionTok{plot}\NormalTok{(Top10UserFarben\_Netzwerk,}
        \CommentTok{\#Layout des Graphen }
        \AttributeTok{layout =}\NormalTok{ layout.fruchterman.reingold,}
        \CommentTok{\#Kanten Dicke und Farbe der Kanten bestimmen}
        \AttributeTok{edge.width =} \FunctionTok{E}\NormalTok{(Top10UserFarben\_Netzwerk)}\SpecialCharTok{$}\NormalTok{gewichtung}\SpecialCharTok{/}\DecValTok{4}\NormalTok{,}
        \AttributeTok{edge.color =} \FunctionTok{E}\NormalTok{(Top10UserFarben\_Netzwerk)}\SpecialCharTok{$}\NormalTok{pixel\_color,}
        \CommentTok{\#Knoten Farbe und Größe bestimmen}
        \AttributeTok{vertex.frame.color =} \StringTok{"black"}\NormalTok{,}
        \AttributeTok{vertex.label.color =} \StringTok{"black"}\NormalTok{,}
        \AttributeTok{vertex.color =}\NormalTok{ vertex\_colors)}
\end{Highlighting}
\end{Shaded}

\includegraphics{SNA-Abgabe_files/figure-latex/unnamed-chunk-16-1.pdf}

Bei der Betrachtung des Netzwerks fallen einige Farben besonders auf.
Die Knoten Blau, Schwarz und Weiß(In der Abbildung als Pink dargestellt)
sind besonders ausgeprägt und bilden in der Mitte ein Dreieck. Der
Knoten Blau ist mit den Usern 14523, 373857 und 17772 sehr stark
verbunden und hat breite Kanten. Gleichzeitig hat der User 14523 eine
zweite Kante zum Knoten Schwarz, welches auch auffällig ist. Der Knoten
Schwarz hat insgesamt 4 stark ausgeprägt Kanten, wobei die stärkste
Kante zum User 183449 führt. Der User 183449 hat meistens die Farben
Weiß und Schwarz genutzt, wobei es auch zur Nutzung von Violett kam.
Auch der Knoten Weiß hat 4 stak ausgeprägte Kanten. Einer dieser Kanten
führt zum User 17772. Dieser User hat eine sehr breite Auswahl an Farben
und hat gleichzeitig sehr ausgeprägt Kanten. Der User hat viele
verschiedene Farben häufig genutzt zu haben. Dieser User fällt damit
sehr auf und ist im Vergleich zu den anderen Usern in der Mitte zu
verorten. Eine weitere Auffälligkeit ist der Knoten Orange. Dieser
Knoten hat vier Kanten, aber diese sind dünn.

\subsection{Untersuchung der Beziehung der Top10 User im Kontext der
Koordinaten}\label{untersuchung-der-beziehung-der-top10-user-im-kontext-der-koordinaten}

Im nächsten Schritt sollen die Top10 User auf ihre Beziehungen
untersucht werden. Es soll herausgefunden werden, ob es Auffälligkeiten
gibt, zwischen den Nutzern und den Koordinaten, die sie ausgewählt
haben. Hierfür passen wir den Subset Top10User an, sodass dieser nur
noch aus den Spalten user und die Koordinaten besteht. Da die
Koordinaten aus den zwei Spalten x und y besteht, werden diese Spalten
zusammengefügt.

\begin{Shaded}
\begin{Highlighting}[]
\CommentTok{\#Neuer Subset auf Grundlage von Top10User}
\NormalTok{Top10UserKoordinaten }\OtherTok{\textless{}{-}}\NormalTok{ Top10User}

\CommentTok{\#Füge eine neue Spalte hinzu, welcher die Koordinatenpaare zusammenfasst}
\NormalTok{Top10UserKoordinaten }\OtherTok{\textless{}{-}} \FunctionTok{transform}\NormalTok{(Top10UserKoordinaten, }\AttributeTok{x\_y =} \FunctionTok{paste}\NormalTok{(x, y, }\AttributeTok{sep =} \StringTok{"\_"}\NormalTok{))}

\CommentTok{\#Die doppelten Werte von x\_y werden zusammengefasst und gezählt und in der }
\CommentTok{\#Spalte count gespeichert und anschließend nach der Anzahl der }
\CommentTok{\# Koordinatenpaare absteigend sortiert.}
\NormalTok{Top10UserKoordinaten }\OtherTok{\textless{}{-}}\NormalTok{ Top10UserKoordinaten }\SpecialCharTok{\%\textgreater{}\%} \FunctionTok{group\_by}\NormalTok{(user, x\_y) }\SpecialCharTok{\%\textgreater{}\%} 
  \FunctionTok{summarise}\NormalTok{(}\AttributeTok{count =} \FunctionTok{n}\NormalTok{()) }\SpecialCharTok{\%\textgreater{}\%} \FunctionTok{arrange}\NormalTok{(}\FunctionTok{desc}\NormalTok{(count))}
\end{Highlighting}
\end{Shaded}

\begin{verbatim}
## `summarise()` has grouped output by 'user'. You can override using the
## `.groups` argument.
\end{verbatim}

\begin{Shaded}
\begin{Highlighting}[]
\CommentTok{\#Anzeigen der Spalten von Top10UserKoordinaten zum testen}
\CommentTok{\#colnames(Top10UserKoordinaten)}
\CommentTok{\#Top10UserKoordinaten}


\CommentTok{\#Erstelen von Vektoren für die Knoten und Kanten des Graphen}
\NormalTok{usernamen2 }\OtherTok{\textless{}{-}} \FunctionTok{c}\NormalTok{(Top10UserKoordinaten}\SpecialCharTok{$}\NormalTok{user)}
\NormalTok{koordinaten }\OtherTok{\textless{}{-}} \FunctionTok{c}\NormalTok{(Top10UserKoordinaten}\SpecialCharTok{$}\NormalTok{x\_y)}
\NormalTok{gewichtung2 }\OtherTok{\textless{}{-}} \FunctionTok{c}\NormalTok{(Top10UserKoordinaten}\SpecialCharTok{$}\NormalTok{count)}

\CommentTok{\#Erstellen einer Matrix aus den Vektoren}
\NormalTok{Top10UserKoordinaten\_Matrix }\OtherTok{\textless{}{-}} \FunctionTok{cbind}\NormalTok{(usernamen2, koordinaten)}
 
\CommentTok{\#Testen ob die Matrix erstellt worden ist}
\CommentTok{\#Top10UserKoordinaten\_Matrix}

\CommentTok{\#Erstellen eines Graphen aus dem Dataframe (Wie in der Vorleusung gezeigt)}
\NormalTok{Top10UserKoordinaten\_Netzwerk }\OtherTok{\textless{}{-}} \FunctionTok{graph\_from\_data\_frame}\NormalTok{(Top10UserKoordinaten\_Matrix, }
  \AttributeTok{directed =} \ConstantTok{FALSE}\NormalTok{)}

\CommentTok{\#Hinzufügen von Attributen zu den Knoten und Kanten}
\NormalTok{Top10UserKoordinaten\_Netzwerk }\OtherTok{\textless{}{-}} \FunctionTok{set\_edge\_attr}\NormalTok{(Top10UserKoordinaten\_Netzwerk, }
  \StringTok{"gewichtung"}\NormalTok{, }\AttributeTok{value =}\NormalTok{ gewichtung2)}

\CommentTok{\#Plotten des Graphen}
\FunctionTok{plot}\NormalTok{(Top10UserKoordinaten\_Netzwerk,}
        \CommentTok{\#Layout des Graphen }
        \AttributeTok{layout =}\NormalTok{ layout.fruchterman.reingold,}
        \CommentTok{\#Kanten Dicke und Farbe der Kanten bestimmen}
        \AttributeTok{edge.width =} \FunctionTok{E}\NormalTok{(Top10UserKoordinaten\_Netzwerk)}\SpecialCharTok{$}\NormalTok{gewichtung,}
        \AttributeTok{edge.color =} \StringTok{"black"}\NormalTok{,}
        \AttributeTok{vertex.label =}\ConstantTok{NA}\NormalTok{)}
\end{Highlighting}
\end{Shaded}

\includegraphics{SNA-Abgabe_files/figure-latex/unnamed-chunk-17-1.pdf}

Aus dem Graphen ist erkenntlich, dass es nur einen Knoten gibt, welcher
zwei User verbindet. Die restlichen Knoten sind alle mit nur einem User
verbunden. Daher sollten die Bereiche des Koordinatensystems in Bereiche
eingeteilt werden. Hierzu wird das Koordinatensystem in 16 Bereiche
eingeteilt. Die X-Achse und die Y-Achse werden jeweils in 4 Bereiche
eingeteilt, sodass 4x4 = 16 Bereiche entstehen. Auf der X-Achse geht ein
Bereich von -500 bis -250, -250 bis 0, 0 bis 250 und 250 bis 500. Auf
der Y-Achse geht ein Bereich von -500 bis -250, -250 bis 0, 0 bis 250
und 250 bis 500. Die Bereiche werden nummeriert mit 1 bis 16 beginnend
von links unten nach rechts oben.

\begin{Shaded}
\begin{Highlighting}[]
\CommentTok{\#Neuer Subset auf Grundlage von Top10User}
\NormalTok{Top10UserKoordinaten }\OtherTok{\textless{}{-}}\NormalTok{ Top10User}

\CommentTok{\#Füge eine neue Spalte hinzu, welcher die Koordinatenpaare zusammenfasst}
\NormalTok{Top10UserKoordinaten }\OtherTok{\textless{}{-}} \FunctionTok{transform}\NormalTok{(Top10UserKoordinaten, }\AttributeTok{x\_y =} \FunctionTok{paste}\NormalTok{(x, y, }\AttributeTok{sep =} \StringTok{"\_"}\NormalTok{))}

\CommentTok{\#Nun werden die Bereiche anhand der x und y Werte bestimmmt.}
\CommentTok{\#Hierfür wird eine neue Spalte erstellt, welche die Bereiche enthält.}
\NormalTok{Top10UserKoordinaten}\SpecialCharTok{$}\NormalTok{Bereich }\OtherTok{\textless{}{-}} \FunctionTok{ifelse}\NormalTok{(Top10UserKoordinaten}\SpecialCharTok{$}\NormalTok{x }\SpecialCharTok{\textless{}} \SpecialCharTok{{-}}\DecValTok{250} \SpecialCharTok{\&}
\NormalTok{  Top10UserKoordinaten}\SpecialCharTok{$}\NormalTok{y }\SpecialCharTok{\textless{}} \SpecialCharTok{{-}}\DecValTok{250}\NormalTok{, }\DecValTok{1}\NormalTok{,}
  \FunctionTok{ifelse}\NormalTok{(Top10UserKoordinaten}\SpecialCharTok{$}\NormalTok{x }\SpecialCharTok{\textless{}} \DecValTok{0} \SpecialCharTok{\&}\NormalTok{ Top10UserKoordinaten}\SpecialCharTok{$}\NormalTok{y }\SpecialCharTok{\textless{}} \SpecialCharTok{{-}}\DecValTok{250}\NormalTok{, }\DecValTok{2}\NormalTok{,}
  \FunctionTok{ifelse}\NormalTok{(Top10UserKoordinaten}\SpecialCharTok{$}\NormalTok{x }\SpecialCharTok{\textless{}} \DecValTok{250} \SpecialCharTok{\&}\NormalTok{ Top10UserKoordinaten}\SpecialCharTok{$}\NormalTok{y }\SpecialCharTok{\textless{}} \SpecialCharTok{{-}}\DecValTok{250}\NormalTok{, }\DecValTok{3}\NormalTok{,}
  \FunctionTok{ifelse}\NormalTok{(Top10UserKoordinaten}\SpecialCharTok{$}\NormalTok{x }\SpecialCharTok{\textless{}} \DecValTok{500} \SpecialCharTok{\&}\NormalTok{ Top10UserKoordinaten}\SpecialCharTok{$}\NormalTok{y }\SpecialCharTok{\textless{}} \SpecialCharTok{{-}}\DecValTok{250}\NormalTok{, }\DecValTok{4}\NormalTok{,}
  \FunctionTok{ifelse}\NormalTok{(Top10UserKoordinaten}\SpecialCharTok{$}\NormalTok{x }\SpecialCharTok{\textless{}} \SpecialCharTok{{-}}\DecValTok{250} \SpecialCharTok{\&}\NormalTok{ Top10UserKoordinaten}\SpecialCharTok{$}\NormalTok{y }\SpecialCharTok{\textless{}} \DecValTok{0}\NormalTok{, }\DecValTok{5}\NormalTok{,}
  \FunctionTok{ifelse}\NormalTok{(Top10UserKoordinaten}\SpecialCharTok{$}\NormalTok{x }\SpecialCharTok{\textless{}} \DecValTok{0} \SpecialCharTok{\&}\NormalTok{ Top10UserKoordinaten}\SpecialCharTok{$}\NormalTok{y }\SpecialCharTok{\textless{}} \DecValTok{0}\NormalTok{, }\DecValTok{6}\NormalTok{,}
  \FunctionTok{ifelse}\NormalTok{(Top10UserKoordinaten}\SpecialCharTok{$}\NormalTok{x }\SpecialCharTok{\textless{}} \DecValTok{250} \SpecialCharTok{\&}\NormalTok{ Top10UserKoordinaten}\SpecialCharTok{$}\NormalTok{y }\SpecialCharTok{\textless{}} \DecValTok{0}\NormalTok{, }\DecValTok{7}\NormalTok{,}
  \FunctionTok{ifelse}\NormalTok{(Top10UserKoordinaten}\SpecialCharTok{$}\NormalTok{x }\SpecialCharTok{\textless{}} \DecValTok{500} \SpecialCharTok{\&}\NormalTok{ Top10UserKoordinaten}\SpecialCharTok{$}\NormalTok{y }\SpecialCharTok{\textless{}} \DecValTok{0}\NormalTok{, }\DecValTok{8}\NormalTok{,}
  \FunctionTok{ifelse}\NormalTok{(Top10UserKoordinaten}\SpecialCharTok{$}\NormalTok{x }\SpecialCharTok{\textless{}} \SpecialCharTok{{-}}\DecValTok{250} \SpecialCharTok{\&}\NormalTok{ Top10UserKoordinaten}\SpecialCharTok{$}\NormalTok{y }\SpecialCharTok{\textless{}} \DecValTok{250}\NormalTok{, }\DecValTok{9}\NormalTok{,}
  \FunctionTok{ifelse}\NormalTok{(Top10UserKoordinaten}\SpecialCharTok{$}\NormalTok{x }\SpecialCharTok{\textless{}} \DecValTok{0} \SpecialCharTok{\&}\NormalTok{ Top10UserKoordinaten}\SpecialCharTok{$}\NormalTok{y }\SpecialCharTok{\textless{}} \DecValTok{250}\NormalTok{, }\DecValTok{10}\NormalTok{,}
  \FunctionTok{ifelse}\NormalTok{(Top10UserKoordinaten}\SpecialCharTok{$}\NormalTok{x }\SpecialCharTok{\textless{}} \DecValTok{250} \SpecialCharTok{\&}\NormalTok{ Top10UserKoordinaten}\SpecialCharTok{$}\NormalTok{y }\SpecialCharTok{\textless{}} \DecValTok{250}\NormalTok{, }\DecValTok{11}\NormalTok{,}
  \FunctionTok{ifelse}\NormalTok{(Top10UserKoordinaten}\SpecialCharTok{$}\NormalTok{x }\SpecialCharTok{\textless{}} \DecValTok{500} \SpecialCharTok{\&}\NormalTok{ Top10UserKoordinaten}\SpecialCharTok{$}\NormalTok{y }\SpecialCharTok{\textless{}} \DecValTok{250}\NormalTok{, }\DecValTok{12}\NormalTok{,}
  \FunctionTok{ifelse}\NormalTok{(Top10UserKoordinaten}\SpecialCharTok{$}\NormalTok{x }\SpecialCharTok{\textless{}} \SpecialCharTok{{-}}\DecValTok{250} \SpecialCharTok{\&}\NormalTok{ Top10UserKoordinaten}\SpecialCharTok{$}\NormalTok{y }\SpecialCharTok{\textless{}} \DecValTok{500}\NormalTok{, }\DecValTok{13}\NormalTok{,}
  \FunctionTok{ifelse}\NormalTok{(Top10UserKoordinaten}\SpecialCharTok{$}\NormalTok{x }\SpecialCharTok{\textless{}} \DecValTok{0} \SpecialCharTok{\&}\NormalTok{ Top10UserKoordinaten}\SpecialCharTok{$}\NormalTok{y }\SpecialCharTok{\textless{}} \DecValTok{500}\NormalTok{, }\DecValTok{14}\NormalTok{,}
  \FunctionTok{ifelse}\NormalTok{(Top10UserKoordinaten}\SpecialCharTok{$}\NormalTok{x }\SpecialCharTok{\textless{}} \DecValTok{250} \SpecialCharTok{\&}\NormalTok{ Top10UserKoordinaten}\SpecialCharTok{$}\NormalTok{y }\SpecialCharTok{\textless{}} \DecValTok{500}\NormalTok{, }\DecValTok{15}\NormalTok{,}
  \FunctionTok{ifelse}\NormalTok{(Top10UserKoordinaten}\SpecialCharTok{$}\NormalTok{x }\SpecialCharTok{\textless{}} \DecValTok{500} \SpecialCharTok{\&}\NormalTok{ Top10UserKoordinaten}\SpecialCharTok{$}\NormalTok{y }\SpecialCharTok{\textless{}} \DecValTok{500}\NormalTok{, }\DecValTok{16}\NormalTok{, }\DecValTok{0}\NormalTok{))))))))))))))))}

\CommentTok{\#Anzeigen der Spalten von Top10UserKoordinaten zum testen}
\CommentTok{\#colnames(Top10UserKoordinaten)}

\CommentTok{\#Die doppelten Werte von Bereich und user werden zusammengefasst und gezählt}
\CommentTok{\#und in der Spalte count gespeichert und anschließend nach der}
\CommentTok{\# Anzahl der Koordinatenpaare absteigend sortiert.}
\NormalTok{Top10UserKoordinaten }\OtherTok{\textless{}{-}}\NormalTok{ Top10UserKoordinaten }\SpecialCharTok{\%\textgreater{}\%} \FunctionTok{group\_by}\NormalTok{(user, Bereich) }\SpecialCharTok{\%\textgreater{}\%}
  \FunctionTok{summarise}\NormalTok{(}\AttributeTok{count =} \FunctionTok{n}\NormalTok{()) }\SpecialCharTok{\%\textgreater{}\%} \FunctionTok{arrange}\NormalTok{(}\FunctionTok{desc}\NormalTok{(count))}
\end{Highlighting}
\end{Shaded}

\begin{verbatim}
## `summarise()` has grouped output by 'user'. You can override using the
## `.groups` argument.
\end{verbatim}

\begin{Shaded}
\begin{Highlighting}[]
\CommentTok{\#Testen ob die Spalte Bereich hinzugefügt worden ist}
\CommentTok{\#Top10UserKoordinaten}

\CommentTok{\#Erstelen von Vektoren für die Knoten und Kanten des Graphen}
\NormalTok{usernamen3 }\OtherTok{\textless{}{-}} \FunctionTok{c}\NormalTok{(Top10UserKoordinaten}\SpecialCharTok{$}\NormalTok{user)}
\NormalTok{bereich }\OtherTok{\textless{}{-}} \FunctionTok{c}\NormalTok{(Top10UserKoordinaten}\SpecialCharTok{$}\NormalTok{Bereich)}
\NormalTok{gewichtung3 }\OtherTok{\textless{}{-}} \FunctionTok{c}\NormalTok{(Top10UserKoordinaten}\SpecialCharTok{$}\NormalTok{count)}


\CommentTok{\#Erstellen einer Matrix aus den Vektoren}
\NormalTok{Top10UserKoordinaten\_Matrix }\OtherTok{\textless{}{-}} \FunctionTok{cbind}\NormalTok{(usernamen3, bereich)}

\CommentTok{\#Top10UserKoordinaten\_Matrix}

\CommentTok{\#Erstellen eines Graphen aus dem Dataframe (Wie in der Vorleusung gezeigt)}
\NormalTok{Top10UserKoordinaten\_Netzwerk }\OtherTok{\textless{}{-}} \FunctionTok{graph\_from\_data\_frame}\NormalTok{(Top10UserKoordinaten\_Matrix, }
  \AttributeTok{directed =} \ConstantTok{FALSE}\NormalTok{)}

\CommentTok{\#Hinzufügen von Attributen zu den Knoten und Kanten}
\NormalTok{Top10UserKoordinaten\_Netzwerk }\OtherTok{\textless{}{-}} \FunctionTok{set\_edge\_attr}\NormalTok{(Top10UserKoordinaten\_Netzwerk, }
  \StringTok{"gewichtung"}\NormalTok{, }\AttributeTok{value =}\NormalTok{ gewichtung3)}



\CommentTok{\#Plotten des Graphen}
\FunctionTok{plot}\NormalTok{(Top10UserKoordinaten\_Netzwerk,}
        \CommentTok{\#Layout des Graphen }
        \AttributeTok{layout =}\NormalTok{ layout.auto,}
        \CommentTok{\#Kanten Dicke und Farbe der Kanten bestimmen}
        \AttributeTok{edge.width =} \FunctionTok{E}\NormalTok{(Top10UserKoordinaten\_Netzwerk)}\SpecialCharTok{$}\NormalTok{gewichtung}\SpecialCharTok{/}\DecValTok{4}\NormalTok{,}
        \CommentTok{\#Zufällige Farben fü die Edges auswählen, um die Kanten besser zu unterscheiden}
        \AttributeTok{edge.color =} \FunctionTok{sample}\NormalTok{(}\FunctionTok{colors}\NormalTok{(), }\FunctionTok{ecount}\NormalTok{(Top10UserKoordinaten\_Netzwerk)),}
        \AttributeTok{vertex.color =} \FunctionTok{ifelse}\NormalTok{(}\FunctionTok{as.numeric}\NormalTok{(}\FunctionTok{V}\NormalTok{(Top10UserKoordinaten\_Netzwerk)}\SpecialCharTok{$}\NormalTok{name)}
          \SpecialCharTok{\textless{}=} \DecValTok{16}\NormalTok{, }\StringTok{"red"}\NormalTok{, }\StringTok{"blue"}\NormalTok{),}
        \AttributeTok{vertex.label.color =} \StringTok{"white"}\NormalTok{,}
        \AttributeTok{vertex.label.size =} \FloatTok{0.5}\NormalTok{)}
\end{Highlighting}
\end{Shaded}

\includegraphics{SNA-Abgabe_files/figure-latex/unnamed-chunk-18-1.pdf}

Dieses neue Netzwerk zeigt deutlich besser die Beziehungen der User zu
den Bereichen. Der User 183449 hat sehr starke Beziehungen zum Bereich
14. Des Weiteren ist der Bereich 14 auch sehr stark mit User 17772
verknüpft. Das bedeutet, dass diese beiden User höchstwahrscheinlich
gegeneinander konkurriert oder zusammengearbeitet haben. Eine weitere
Auffälligkeit ist, dass der User 17772 weitere sehr starke Beziehungen
zu den Bereichen 12 und 16, neben dem Bereich 14 hat. Es wurde bereits
vorher ermittelt, dass dieser User die meisten Pixel gefärbt hat.
Außerdem gibt es manche Bereiche, die nur von einem User alleine gefärbt
worden sind. Hierzu kann der Bereich 11 und 14523 betrachtet werden. Der
Brereich 7 wurde nur von einem einzigen User (395737) gefärbt. Dieser
User hatte also nicht so viel User-Interaktionen, wie die anderen User.
In der Mitte befinden sich 4 weitere User die relativ nah aneinander
sind und sehr viele kleinere Verbindungen zu den Bereichen haben. Diese
User haben wohl sich auf viele unterschiedliche Bereiche verteilt und
nicht nur einen bestimmten Bereich gefärbt. Daraus könnte man vermuten,
dass diese User als Unterstützer tätig waren. Zusätzlich könnte man
vermuten, dass die User mit starken Beziehungen in einem Bereich als
Gruppenführer agiert haben.

\section{Interpretation der Ergebnisse und Beantwortung der
Forschungsfrage}\label{interpretation-der-ergebnisse-und-beantwortung-der-forschungsfrage}

\subsection{Interpretation der
Ergebnisse}\label{interpretation-der-ergebnisse}

Es wurde ein guter Eindruck geschaffen über die Verteilung der Pixel und
die Verteilung der Farben. Generell sind viele Indizien für Beziehungen
und Abhängigkeiten zwischen den Usern zu erkennen. Allerdings ist die
Bestätigung dieser Indizien nicht möglich, da die User pseudonymisiert
sind und unbekannt. Daher können nur Vermutungen angestellt werden.

\subsection{Beantwortung der
Forschungsfrage}\label{beantwortung-der-forschungsfrage}

Die zu Beginn gestellten Forschungsfragen sollen nun beantwortet werden.

\begin{itemize}
\tightlist
\item
  Sind Muster zu erkennen und in welchen Bereichen der Leinwand sind
  diese Muster zu erkennen?
\end{itemize}

Auf der Leinwand sind viele Muster zu erkennen. Das soziale Experiment
hat gezeigt, dass die Teilnehmer einander beeinflussen, aber dennoch
Ideen von Bildern und Mustern umgesetzt werden können. Diese Muster sind
auf der Leinwand zu erkennen.

\begin{itemize}
\tightlist
\item
  Welche Bereiche der Leinwand wurden besonders oft gefüllt?
\end{itemize}

Die Visualisierung mit der Heatmap veranschaulicht sehr gut die
Bereiche, welche sehr oft gefüllt worden sind. Diese Bereiche werden in
rot dargestellt.

\begin{itemize}
\tightlist
\item
  Welche Farben haben die aktivsten Nutzer besonders oft genutzt?
\end{itemize}

Die Farben der aktivsten Nutzer sind sehr unterschiedlich, aber es gibt
einige Farben, welche besonders oft genutzt worden sind. Die Farbe
Blau,Schwarz und Weiß (In der Abbildung als Pink dargestellt) sind
besonders oft genutzt worden.

\begin{itemize}
\tightlist
\item
  Welche Bereiche haben die aktivsten Nutzer besonders oft gefüllt?
\end{itemize}

Die Bereiche, welche die aktivsten Nutzer besonders oft gefüllt haben,
sind die Bereiche 14, 12 und 16.

\begin{itemize}
\tightlist
\item
  Gibt es generell Beziehungen zwischen den Nutzern?
\end{itemize}

Generell können Beziehungen zwischen den Nutzern erkannt werden,
allerdings sind dies nur Indizien. Eine eindeutige Bestätigung ist nicht
möglich.

\subsection{Fazit}\label{fazit}

Die Social Network Analysis ist eine sehr interessante Methode, um
Beziehungen zwischen Usern zu untersuchen. Es wurden ein besonders
interessanter Datensatz ausgewählt, welcher sehr viele Möglichkeiten zur
Untersuchung bietet. Aus den gegebenen Daten konnte eine gute Analyse
erstellt werden, welche viele Indizien für Beziehungen zwischen den
Usern aufzeigt. Weitere Visualisierungen und Untersuchungen sind
möglich, aber würden den Rahmen der Arbeit übersteigen. Dies ist ein
kurzer Einblick in die Social Network Analysis und die Möglichkeiten,
welche diese bietet.

\end{document}
